\chapter{Robot Construction}


\subsection{Hardware}

some HW examples, schematic, photos

\begin{figure}[!htb]
    \centering
    \includegraphics[scale=0.1]{../images/robot/robot_02.jpg}
    \caption{Motoko Ice Dragon X robot}
    \label{fig:line_following_robot}
\end{figure}

\begin{figure}[!htb]
    \centering
    \includegraphics[scale=0.075]{../images/robot/robot_03.jpg}
    \caption{Motoko Ice Dragon X robot}
    \label{fig:line_following_robot_b}
\end{figure}

\newpage
\subsection{Sensors}

sensor chosing

\newpage
\subsection{Motors}


\subsubsection{BLDC motor}

Brushless motors with electronic commutatio is state of the art motor technology.
Especially with feeedback (usually magnetic encoder) and field oriented control (FOC),
it allows to achieve full torque, maximal efficiency and very precission regulating, 
all : torque, velocity and position.

However, for small racing robots, there are not available brushless motors with encoder.
This must be solved standalone - with custom diametrically magnetised magnet, and simple 3D 
print.




The out runner motor (mostly used in drones), can be modified for running wheel 
directly. In the figure \ref{fig:original_1404_motor} is disassembled motor 
of size 1404 (14mm diameter, 4mm stator height).
Such a small motor for drone is designed for high speed, and low torque, 
we can see there is only few turns or relative thick wire.

\begin{figure}[!htb]
    \centering
    \includegraphics[scale=0.1]{../images/motors/20241108_101700.jpg}
    \caption{Original 1404 motor}
    \label{fig:original_1404_motor}
\end{figure}


\subsubsection{Ready to use motor}

Good choice is gimbal motor, e.g. in size 2204, with 260kV. Motor is sufficient slow,
however having relativly high weight.


\subsubsection{Custom winding}

The original drone motors are designed for high speed (RPM, high kV number), and low torque,
operating with huge current and low voltage. This power setup have historical background,
where in RC-world was hard to obtain higher voltage from old NiCd cells.
For robot applications, we are looking for higher torque, low RPM, higher voltage is 
mostly wellocmed, leading to smaller drivers, lower heath producing.

In the figure \ref{fig:original_disassembled_motor} is original winding of 1404 motor,
each pole cotains 12 turns on stator, stator have 12 poles (4 for each phase), and in total 14 magnetic poles rotors. 

\begin{figure}[!htb]
    \centering
    \includegraphics[scale=0.1]{../images/motors/20241108_101700.jpg}
    \caption{Original disassembled motor}
    \label{fig:original_disassembled_motor}
\end{figure}

In the figure \ref{fig:winding_diagram} is generic winding diagram for generic 
mosty used 12 poles stator. The wire with 0.15mm is good choice for small robot, 
and it is possible to fit 35 turns for each pole. This motor draws arround 2..3amps 
from 2cell LiPo accumulator.
Winding can be done by hand with some patiance and time. 

The best tool to obtain exact diagram for given stator and number magnets can be find here
\href{https://www.bavaria-direct.co.za/scheme/calculator/}{https://www.bavaria-direct.co.za/scheme/calculator/} 


\begin{figure}[!htb]
    \centering
    \includegraphics[scale=0.4]{../images/motors/motor_winding.jpg}
    \caption{Three phase winding diagram for 12 poles}
    \label{fig:winding_diagram}
\end{figure}

Rewinded motor can be seen in the figure \ref{fig:rewinded_motor}.


\begin{figure}[!htb]
    \centering
    \includegraphics[scale=0.3]{../images/motors/rewinded.png}
    \caption{Rewinded motor}
    \label{fig:rewinded_motor}
\end{figure}

Next step is to add encoder. We will follow best practices, we use diametrically magnetised 
magnet, and magnetic i2c sensor. The magnet is not glued directly - metal to metal is 
not strognest, thefore tiny 3D cylinder is used for better holding the magnet, 
as shown in the figure \ref{fig:encoder_magnet_mount}.

\begin{figure}[!htb]
    \centering
    \includegraphics[scale=0.1]{../images/motors/20241110_183215.jpg}
    \caption{Encoder magnet mount}
    \label{fig:encoder_magnet_mount}
\end{figure}

The motor is held by simple 3D printed bracket, with 4 screws on motor, and 2 mount screws.
As seen on figure \ref{fig:motor_mout} and encoder place for SO8 SMD encoder in figure 
\ref{fig:motor_mout_encoder}. This overall summary shows how to modify existing out runner 
motor for purpose as wheel.


\begin{figure}[!htb]
    \centering
    \includegraphics[scale=0.1]{../images/motors/20241024_211909.jpg}
    \caption{Motor mount}
    \label{fig:motor_mout}
\end{figure}

\begin{figure}[!htb]
    \centering
    \includegraphics[scale=0.1]{../images/motors/20241025_132703.jpg}
    \caption{Motor mount with encoder}
    \label{fig:motor_mout_encoder}
\end{figure}